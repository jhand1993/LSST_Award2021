\documentclass[modern]{aastex63}
\usepackage[utf8]{inputenc} 
\usepackage{hyperref}
\usepackage{listings}
\usepackage{fancyhdr}   
\usepackage{geometry}
\usepackage{amsmath}
\renewcommand{\baselinestretch}{1.0}
\usepackage{amssymb}
%\renewcommand{rnaas}{True}

\geometry{lmargin=1in, bmargin=1in, tmargin=0.5in, rmargin=1in}

\pagestyle{plain}

\begin{document}
\title{Dust Attenuation and Type Ia Supernova Host Bias}


\author{Michael Wood-Vasey}
\affiliation{University of Pittsburgh, \textnormal{wmwv@pitt.edu}}
%\affiliation{\textbf{\textnormal{wmwv@pitt.edu}}}

\author{Jared Hand}
\affiliation{University of Pittsburgh, \textnormal{jsh89@pitt.edu}}
%\affiliation{\textbf{\textnormal{jsh89@pitt.edu}}}

\begin{abstract}
    Building on our research group's past work on the type Ia supernova (SN~Ia) host bias, we wish to supplement graduate student Jared Hand and fund an undergraduate in a project studying dust attenuation corrections' influences on SN~Ia host bias corrections and its connection to SN~Ia color variation. Better understanding SN~Ia color variation and characteristics of the host bias will improve the utility of SN~Ia for measuring dark energy.
    \begin{itemize}
        \item \$10,000 requested
    \end{itemize}
\end{abstract}


\section{Personnel and Proposed Project}
We request support for graduate student Jared Hand and one undergraduate student under Professor Michael Wood-Vasey at the University of Pittsburgh for our project exploring the relationship between host galaxy dust effects and the measurement of the SN~Ia host bias.
Graduate student Hand, using this support to cover a gap in funding, will build the proposed project's foundations:
\begin{itemize}
    \item Developing a proper hierarchical Bayesian model with the statistics programming language Stan\footnote{\url{https://mc-stan.
    org/}} for our proposed project.
    \item Deploying a SQlite database and interact with it using SQLAlchemy\footnote{\url{https://www.sqlalchemy.org/}}.
    \item Developing a framework to properly integrate differing galaxy dust attenuation corrections into our analysis and interpreting their impact.
\end{itemize}
With Professor Wood-Vasey in support, graduate student Hand will work closely with the undergraduate in developing the following skills:
\begin{itemize}
    \item Scalable, data-driven programming techniques using Python and github.
    \item Bayesian statistics using a hands-on approach with astronomical data. 
    \item Introductory relational database querying using a local project SQLite database.
\end{itemize}
Data-driven analysis skills will be developed through manipulation of large SN~Ia data sets presented within the context of LSST's science goals.
Graduate student Hand and the undergraduate will both present project results and their respective contributions to DESC and PCW, providing practice in science communication and exposure to the LSST science community.
This project builds upon graduate student Hand's prior research \citep{Hand2021} and will contribute towards their final dissertation.

SN~Ia are standardized using correlations between peak brightness and both color and explosion duration to reduce the naturally low scatter in peak brightness.
There is also an established bias between SN~Ia properties and the properties of its host galaxies that propagates through SN~Ia standardization procedure, resulting in the well-studied mass step \citep{Sullivan2010}. 
Studies have found that linear combinations of stellar age or star formation rate (SFR) with stellar mass better account for both the observed host bias and its propagation through standardization \citep{Rigault2018,Rose2021}.
Stellar age and SFR are strongly degenerate with host dust properties, though, and~\cite{Brout2021} found that although SN~Ia color is not entirely explained by a dust-only model, a more sophisticated standardization model incorporating host dust effects accounts for the mass step.
We propose exploring the effects of host dust correction techniques on the measured host bias to help bridge a gap in knowledge between SN~Ia color variation from dust and the effects of dust on the host bias.

We took a first step in understanding the influence of observation and fitting techniques on SN~Ia host bias measurements in~\cite{Hand2021} using the Integral Field Spectra (IFS) observations of PISCO SN~Ia host sample \citep{Galbany2018}.  
This proposed project will triple our prior statistics by including AMUSING \citep{Galbany2016a} and instead will focus on host galaxy dust attenuation's effects on differing SFR tracers. 
Using PISCO with the higher resolution AMUSING IFS SN~Ia host samples alongside overlapping UV surveys, we will compare specific SFR (sSFR) estimates from H$\alpha$ flux and UV photometry both with and without dust attenuation corrections.
sSFR is a natural host property to explore given its intrinsic nature, with different observables (H$\alpha$ and UV flux) that trace differing epochs of star formation and by it being a linear combination of host properties: $\log{\rm sSFR} = \log{\rm SFR} - \log{\rm Mass}$.
Along with the Balmer decrement H$\alpha$/H$\beta$ correction, we will implement and compare sophisticated dust attenuation models from~\cite{Salim2018} and~\cite{Narayanan2018}.  
A hierarchical Bayesian framework implemented with Stan will be used to study the effects of differing attenuation techniques on standardization.

\section{Budget Narrative}
We request \$5000 each in funding for graduate student Hand and an undergraduate to supplement their stipends.  We wish to disburse funds directly to our institution, which has an administrative fee.

\bibliographystyle{aasjournal}
\bibliography{main}

\end{document}
